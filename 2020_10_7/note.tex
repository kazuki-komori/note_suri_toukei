\synctex=1
\documentclass[dvipdfmx,10pt, a4j]{jarticle}
%----------------------------------------------------------
%パッケージ読み込み
\usepackage{amsmath}
\usepackage{amssymb}
\usepackage{amsthm} %定理環境の拡張
\usepackage{ascmac}
\usepackage{bm}
\usepackage{cases}
\usepackage{comment} %非表示にするためのコメント
\usepackage{enumerate}
\usepackage{float} %画像をその場に表示.[h]の代わりに[H]
\usepackage{graphicx} % eps 形式の図版取り込みのため
\usepackage{mathrsfs}
\usepackage{url}
\usepackage[dvipdfmx]{hyperref}
%----------------------------------------------------------

%----------------------------------------------------------
%命題関係の定義
\theoremstyle{definition}
\newtheorem{definition}{定義}[section]
\newtheorem{theorem}{定理}[section]
\newtheorem{proposition}[theorem]{命題}
\newtheorem{lemma}[theorem]{補題}
\newtheorem{col}[theorem]{系}
\newtheorem{example}{例}[section]
\newtheorem{remark}{注意}[section]
%----------------------------------------------------------

%タイトル・著者===================================================
\title{第2回 数理統計 レポート}
\author{小森 一輝}
%=================================================================

%本文開始=========================================================
\begin{document}

    \maketitle

%カウンタ--------------------------------------------------
\setcounter{section}{2}
%\setcounter{subsection}{0}
%\setcounter{subsubsection}{0}
%\setcounter{theorem}{0}
%----------------------------------------------------------命題2.2
    \noindent
    \textbf{命題 2.2.} 和事象, 積事象, 余事象\\
    集合 A が事象であれば, その集合 $A^(c)$ も事象となり、\textbf{余事象}と呼ばれる。\\
    また、以下に集合 A, Bそれぞれについての和集合 $A \cup B$, 積集合 $A \cap B$それぞれについて示す。
    %---------------以下証明
    \begin{proof} $A, B \in \mathcal{A}$のとき\\
        \begin{enumerate}
            \renewcommand{\labelenumi}{\roman{enumi})}
            \item \textbf{余事象 $A^{c}$} \\
                $A \in \mathcal{A}$ より、 $A^{c} \in \mathcal{A}$ $(\because 定義2.1(ii))$
            \item \textbf{和事象 $A \cup B$} \\
                $A, B \in \mathcal{A}$ より、 $A \cup B \in \mathcal{A}$ $(\because 定義2.1(iii))$
            \item \textbf{積事象 $A \cap B$} \\
                $A, B \in \mathcal{A}$ より、 $A^{c}, B^{c} \in \mathcal{A}$ $(\because 定義2.1(ii))$\\
                $A^{c}, B^{c} \in \mathcal{A}$ より、 $A^{c} \cup B^{c} \in \mathcal{A}$ $(\because 定義2.1(iii))$\\
                $A^{c} \cup B^{c} \in \mathcal{A}$より、\\
                ド・モルガンの法則を用いると、\\
                $(A^{c}, B^{c})^{c} = A \cup B \in \mathcal{A}$ $(\because 定義2.1(ii))$
        \end{enumerate}
    \end{proof}
    %---------------以下補足
    \subparagraph*{補足}
    補足項目の追記\\
    $A \cap B = \phi のとき、A と B は排反である。$\\

% ------------------------------------------------------------定義2.4
    \noindent
    \textbf{定義 2.4.} 排反事象\\
    事象A, B に対して、 $A \cup B = \phi$ が成り立つとき、事象A と 事象B は互いに排反であるという。\\

% ------------------------------------------------------------定義2.5
    \noindent
    \textbf{定義 2.5.} 可測空間\\
    空でない集合 $\Omega$ とその完全加法族 $\mathcal{A}$ の組$(\Omega, \mathcal{A})$を \textbf{可測空間} と呼ぶ。\\

% ------------------------------------------------------------定義2.9
    \noindent
    \textbf{定義 2.9.} コルモロゴフの確率の公理\\
    $(\Omega, \mathcal{A})$を可測空間とし、P を $\mathcal{A}$ を定義域とする、実数値関数とする。
    \begin{enumerate}
        \renewcommand{\labelenumi}{\roman{enumi})}
        \item 任意の $A \in \mathcal{A}$ に対して、 $0 \geqq P(A) \geqq 1$\\
        このとき、0は空事象、1は全事象を表す。\\
        \item $P(\Omega) = 1$
        \item $\bm{A_i} \in \mathcal{A}(i = 1, 2, \dots), {A_i} \cap {A_j} = \phi (i \neq j)$ ならば, \\
        $P\left(\displaystyle\bigcup_{i=1}^{\infty}{A_i}\right) = \Sigma_{i=1}^{\infty}P(A_i)$
    \end{enumerate}
    以上の3つの条件を満たすとき、Pを確立空間上の \textbf{確率測度} とよぶ。\\
    また、事象 A に対する確率測度 P の値、P(A) を事象 A の \textbf{確率} とよぶ。\\

% ------------------------------------------------------------定義2.10
    \noindent
    \textbf{定義 2.10.} 確率空間\\
    $(\Omega, \mathcal{A}, P)$\\
    上記のように、標本空間 $\bm{\Omega}$, 完全加法族 $\bm{\mathcal{A}}$, 加速度 $\bm{P}$ の組を \textbf{確率空間} という。\\

% ------------------------------------------------------------例2.1
    \noindent
    \textbf{例 2.1.} 古典的確率(ラプラス流)\\
    標本空間を $\Omega$ = \{${\omega_i} \mid i = 1,2,\dots n$\} とする。実数値関数 P を P({$\omega_i$}) = $1/n(i = 1,2,\dots n)$ を満たし、
    かつ、事象 $A(\subset \Omega)$ に対して\\

    $P(A) = \frac{n(A)}{n(\Omega)} = \frac{n(A)}{n}$\\

    を満たすと定義する。このとき、関数 P は確率となる。\\

    $P({\omega_i}) = \frac{1}{n}(i = 1,2,\dots n)$

    上式は「同様に確からしい」状況を表す。\\

    $A \in \mathcal{A} (A \subset \Omega)$\\

    $P(A) = \frac{n(A)}{n(\Omega)}$ (※$ n(A) は A に含まれる個数$)\\

    このように定義すると P は確率測度を表す。\\

% ---------------以下証明
    \begin{proof} 定義2.9(コルモロゴフの確率の公理の証明)\\
    \begin{enumerate}
        \renewcommand{\labelenumi}{\roman{enumi})}
        \item
        $0 \leq \frac{n(A)}{n} \leq \frac{n(\Omega)}{n} = P(\Omega)$
        \item
        $0 \leq P(A) \leq 1   (\because 定義 i より)$
        \item
        ${A_i} \cap {A_j} = P(i \neq j)$\\
        $P\left(\bigcup_{i=1}^{n}{A_i}\right) = \frac{n(\cup {A_i})}{n(\Omega)} = \frac{\sum_{i=1}^n}{n} n({A_i}) = \sum_{i=1}^n \frac{n(A_i)}{n} = \sum_{i=1}^n P(A_i)$
    \end{enumerate}
    \end{proof}
    \begin{itembox}[l]{補足}
        ラプラス流の確率はコルモロゴフの確率の公理を満たす。
    \end{itembox}
    \begin{itembox}[l]{矛盾点\ldots}
        適用範囲nは有限個のみ、 $\infty$ の場合 0 に収束\\
        「同様に確からしい」は等確率?等確率は現実で起こり得ることはないため現実では適用不可
    \end{itembox}\\

% ------------------------------------------------------------定理2.5
    \noindent
    \textbf{定理 2.5.} 確率空間\\
    $(\Omega, \mathcal{A}, P)$\\
    \begin{proof} 証明\\
    \begin{enumerate}
        \renewcommand{\labelenumi}{\roman{enumi})}
% -----------------------------------(i)
        \item
        $P(\Omega) = 0$\\
        \begin{align*}
            \phi = \phi \cup \phi, \phi \cap \phi = \phi より
            P(\phi) = P(\phi \cup \phi) = P(\phi) + P(\phi)\\
            よって、 P(\phi) = 0 \quad (\because P(\phi) \geq 0)
        \end{align*}
% -----------------------------------(ii)
        \item
        $A, B \in \mathcal{A}, A \subset B ならば P(A) \leq P(B))$
        \begin{align*}
            (A, B \mathcal{A}, A \subset B)\\
            B = B \cap \Omega &= B \cap (A \cup A^{c})\\
            &= (B \cap A) \cup (B \cap A^{c}) \\
            &= A \cup (B \cap A^{c}) \\
        \end{align*}
        したがって、 $P(B) = PA \cup (B \cap A^{c}) = P(A) + P(A) + P(B \cap A^{c}) \geq P(A)$
% -----------------------------------(iii)
        \item
        $A \in \mathcal{A} ならば, P(A)+ P(A^{c}) = 1$
        \begin{center}
            $A \cap A^{c} = \phi$\\
            $A \cup A^{c} = \Omega$\\
            $1 = P(\Omega) = P(A \cup A^{c}) = P(A) + P(A^{c}) \quad (\because AとA^{c} は排反)$
        \end{center}
% -----------------------------------(iv)
        \item
        $A, B \in \mathcal{A} ならば, P(A \cup B) = P(A) + P(B) - P(A \cap B)$
        \begin{align}
            P(A \cup B) = P(A \cap B^{c}) + P(A \cap B) + P(A^{c} \cap B)\\
            P(A) = P(A \cap B^{c}) + P(A \cap B)\\
            P(B) = P(A^{c} \cap B) + P(A \cap B)
        \end{align}
        \begin{align*}
            (2)より、 P(A \cap B^{c}) = P(A) - P(A \cap B)\\
            (3)より、 P(A^{c} \cap B) = P(B) - P(A \cap B)\\
            これらを (1) に代入すると示せた。
        \end{align*}
    \end{enumerate}
    \end{proof}
    \newpage
% ------------------------------------------------------------例2.2
    \noindent
    \textbf{例 2.2.} 確率空間の例\\
    表が出る確率が P のコインを投げる。H を表、Tを裏とすると各要素を以下のように表すことができる。\\
    $標本空間 \Omega = \{ H, T \}$\\
    $完全加法族 \mathcal{A} = \{ \Omega, \phi, \{H\}, \{T\} \}$\\

    確率はそれぞれ、
    $P(\Omega) = 1$\\
    $P(\phi) = 0$\\
    $P(\{H\}) = p$\\
    $P(\{H\}^{c}) = 1-p$\\

% ------------------------------------------------------------例2.4
    \noindent
    \textbf{例 2.4.} スターリングの公式\\
    以下の式が成り立つ。\\
    \[
        \lim_{x \to \infty} \frac{n!}{\sqrt{2 \pi n}\left(\frac{n}{e}\right)^{n}} = 1\\
    \]
    つまり、十分大きな $n \in \mathcal{N} について, n!$を以下のように表すことができる。
    \begin{align}
        n! \approx \sqrt{2 \pi n}\left(\frac{n}{e}\right)^{n}
    \end{align}

% ------------------------------------------------------------定理2.6
    \noindent
    \textbf{定理 2.6.} 一般加法定理\\
    和の確率は, 以下のように積の確率で表現することができる。\\
    $A_i \in \mathcal{A} ならば、$\\
    \begin{align*}
        P(\bigcup_{i=1}^{n}{A_i}) = \sum_{i = 1}^n P(A_i) - \sum_{1 < i_1 < i_2 < n} P(A_{i1} \cap A_{i2}) + \sum_{1 < i_1 < i_2 < i_3 < n} P(A_{i1} \cap A_{i2} \cap A_{i3}) \\
        + \dots + (-1)^{l-1} \sum_{1 < i_1 < i_2 \dots < i_l < n} P(A_{i1} \cap A_{i2} \cap \dots \cap A_{il})\\
        + \dots + (-1)^{n-1}P(\bigcup_{i=1}^{n}{A_i})\\
    \end{align*}
    が成り立つ。これを, \textbf{一般加法定理}という。

\end{document}
%本文ここまで=========================================================
