\synctex=1
\documentclass[dvipdfmx,10pt, a4j]{jarticle}
%----------------------------------------------------------
%パッケージ読み込み
\usepackage{amsmath}
\usepackage{amssymb}
\usepackage{amsthm} %定理環境の拡張
\usepackage{ascmac}
\usepackage{bm}
\usepackage{cases}
\usepackage{comment} %非表示にするためのコメント
\usepackage{enumerate}
\usepackage{float} %画像をその場に表示.[h]の代わりに[H]
\usepackage{graphicx} % eps 形式の図版取り込みのため
\usepackage{mathrsfs}
\usepackage{url}
\usepackage[dvipdfmx]{hyperref}
%----------------------------------------------------------

%----------------------------------------------------------
%命題関係の定義
\theoremstyle{definition}
\newtheorem{definition}{定義}[section]
\newtheorem{theorem}{定理}[section]
\newtheorem{proposition}[theorem]{命題}
\newtheorem{lemma}[theorem]{補題}
\newtheorem{col}[theorem]{系}
\newtheorem{example}{例}[section]
\newtheorem{remark}{注意}[section]
%----------------------------------------------------------

%タイトル・著者===================================================
\title{第2回 数理統計 レポート}
\author{小森 一輝}
%=================================================================

%本文開始=========================================================
\begin{document}

    \maketitle

%カウンタ--------------------------------------------------
\setcounter{section}{2}
%\setcounter{subsection}{0}
%\setcounter{subsubsection}{0}
%\setcounter{theorem}{0}
%----------------------------------------------------------命題2.2
    \noindent
    \textbf{命題 2.2.} 和事象, 積事象, 余事象\\
    集合 A が事象であれば, その集合 $A^(c)$ も事象となり、\textbf{余事象}と呼ばれる。\\
    また、以下に集合 A, Bそれぞれについての和集合 $A \cup B$, 積集合 $A \cap B$それぞれについて示す。
    %---------------以下証明
    \begin{proof} $A, B \in \mathcal{A}$のとき\\
        \begin{enumerate}
            \renewcommand{\labelenumi}{\roman{enumi})}
            \item \textbf{余事象 $A^{c}$} \\
                $A \in \mathcal{A}$ より、 $A^{c} \in \mathcal{A}$ $(\because 定義2.1(ii))$
            \item \textbf{和事象 $A \cup B$} \\
                $A, B \in \mathcal{A}$ より、 $A \cup B \in \mathcal{A}$ $(\because 定義2.1(iii))$
            \item \textbf{積事象 $A \cap B$} \\
                $A, B \in \mathcal{A}$ より、 $A^{c}, B^{c} \in \mathcal{A}$ $(\because 定義2.1(ii))$\\
                $A^{c}, B^{c} \in \mathcal{A}$ より、 $A^{c} \cup B^{c} \in \mathcal{A}$ $(\because 定義2.1(iii))$\\
                $A^{c} \cup B^{c} \in \mathcal{A}$より、\\
                ド・モルガンの法則を用いると、\\
                $(A^{c}, B^{c})^{c} = A \cup B \in \mathcal{A}$ $(\because 定義2.1(ii))$
        \end{enumerate}
    \end{proof}
    %---------------以下補足
    \subparagraph*{補足}
    補足項目の追記\\
    $A \cap B = \phi のとき、A と B は排反である。$\\

% ------------------------------------------------------------定義2.4
    \noindent
    \textbf{定義 2.4.} 排反事象\\
    事象A, B に対して、 $A \cup B = \phi$ が成り立つとき、事象A と 事象B は互いに排反であるという。\\

% ------------------------------------------------------------定義2.5
    \noindent
    \textbf{定義 2.5.} 可測空間\\
    空でない集合 $\Omega$ とその完全加法族 $\mathcal{A}$ の組$(\Omega, \mathcal{A})$を \textbf{可測空間} と呼ぶ。\\

% ------------------------------------------------------------定義2.9
    \noindent
    \textbf{定義 2.9.} コルモロゴフの確率の公理\\
    $(\Omega, \mathcal{A})$を可測空間とし、P を $\mathcal{A}$ を定義域とする、実数値関数とする。
    \begin{enumerate}
        \renewcommand{\labelenumi}{\roman{enumi})}
        \item 任意の $A \in \mathcal{A}$ に対して、 $0 \geqq P(A) \geqq 1$\\
        このとき、0は空事象、1は全事象を表す。\\
        \item $P(\Omega) = 1$
        \item $\bm{A_i} \in \mathcal{A}(i = 1, 2, \dots), {A_i} \cap {A_j} = \phi (i \neq j)$ ならば, \\
        $P\left(\displaystyle\bigcup_{i=1}^{\infty}{A_i}\right) = \Sigma_{i=1}^{\infty}P(A_i)$
    \end{enumerate}

% ------------------------------------------------------------定義2.2
\end{document}
%本文ここまで=========================================================
