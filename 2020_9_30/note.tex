\synctex=1
\documentclass[dvipdfmx,10pt, a4j]{jarticle}
%----------------------------------------------------------
%パッケージ読み込み
\usepackage{amsmath}
\usepackage{amssymb}
\usepackage{amsthm} %定理環境の拡張
\usepackage{ascmac}
\usepackage{bm}
\usepackage{cases}
\usepackage{comment} %非表示にするためのコメント
\usepackage{enumerate}
\usepackage{float} %画像をその場に表示.[h]の代わりに[H]
\usepackage{graphicx} % eps 形式の図版取り込みのため
\usepackage{mathrsfs}
\usepackage{url}
\usepackage[dvipdfmx]{hyperref}
%----------------------------------------------------------

%----------------------------------------------------------
%命題関係の定義
\theoremstyle{definition}
\newtheorem{definition}{定義}[section]
\newtheorem{theorem}{定理}[section]
\newtheorem{proposition}[theorem]{命題}
\newtheorem{lemma}[theorem]{補題}
\newtheorem{col}[theorem]{系}
\newtheorem{example}{例}[section]
\newtheorem{remark}{注意}[section]
%----------------------------------------------------------

%タイトル・著者===================================================
\title{第1回 数理統計 レポート}
\author{小森 一輝}
%=================================================================

%本文開始=========================================================
\begin{document}

    \maketitle

%カウンタ--------------------------------------------------
\setcounter{section}{2}
%\setcounter{subsection}{0}
%\setcounter{subsubsection}{0}
%\setcounter{theorem}{0}
%----------------------------------------------------------定義2.1
    \begin{definition} 加法族\\
    空でない集合$\Omega$の部分集合の族、$\mathcal{A}$が以下の条件を満たすとき、\textbf{加法族}と呼ぶ。
    \begin{enumerate}
        \renewcommand{\labelenumi}{\roman{enumi})}
        \item $\Omega \in \mathcal{A}$
        \item $A \in \mathcal{A}$ ならば $A^{c} \in \mathcal{A}$
        \item $A \in \mathcal{A} ,B \in \mathcal{A} $ ならば $A \cup B \in \mathcal{A}$
    \end{enumerate}
    %---------------以下補足
    \subparagraph*{補足}
    補足項目の追記\\
    要素 ($\bm{a}$)と集合 ($\bm{A}$)、$\mathcal{A}$ は以下の関係で表すことができる。\\
    a $\in$ A $\in \mathcal{A}$\\
    \begin{itembox}[l]{未定義用語}
        集合 $\bm{A}$\\
        要素 $\bm{a}$\\
        含まれる $\bm{\in}$
    \end{itembox}
    \end{definition}
% ------------------------------------------------------------定義2.2
    \begin{definition} 完全加法族\\
    空でない集合$\Omega$の部分集合の族、$\mathcal{A}$が以下の条件を満たすとき、\textbf{完全加法族}と呼ぶ。
    \begin{enumerate}
        \renewcommand{\labelenumi}{\roman{enumi})}
        \item $\Omega \in \mathcal{A}$
        \item $\bm{A} \in \mathcal{A}$ ならば $\bm{A^{c}} \in \mathcal{A}$
        \item $\bm{A_i} \in \mathcal{A}(i = 1, 2, \dots)$ ならば $\displaystyle\bigcap_{i=1}^{\infty}{A_i} \in \mathcal{A}$
    \end{enumerate}
    %---------------以下補足
    \subparagraph*{補足}
    補足項目の追記\\
    \begin{itembox}[l]{集合の可算性}
        $P(A \cup B) = A \cap B^{c}$ + $A \cap B$ + $A^{c} \cap B$
    \end{itembox}
    \end{definition}
    \clearpage
% ------------------------------------------------------------定義2.3
    \begin{definition} 事象, 根元事象\\
    完全加法族$\mathcal{A}$の要素は、\textbf{事象}と呼ぶ。つまり、集合$\Omega$ の部分集合が事象である。\\
    また、ただ一つの要素 $\omega$ を持つ集合 $\{ \omega \}$ が事象ならば、$\{ \omega \}$は\textbf{根元事象}と呼ばれる。
    \begin{proposition} 全事象, 空事象\\
        集合 $\Omega$, $\phi$はそれ自身が事象であり、それぞれ\textbf{全事象}, \textbf{空事象} と呼ばれる。
    \end{proposition}
    %---------------以下証明
    \begin{proof}
        定義2.2(i)より、\\
        $\Omega \in \mathcal{A}$\\
        上式と定義2.2(ii)より、\\
        $\Omega \in \mathcal{A} \Rightarrow \Omega^{c} = \phi \in \mathcal{A}$\\
    \end{proof}
    \end{definition}
\end{document}
%本文ここまで=========================================================
