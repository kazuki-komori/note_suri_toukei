\synctex=1
\documentclass[dvipdfmx,10pt, a4j]{jarticle}
%----------------------------------------------------------
%パッケージ読み込み
\usepackage{amsmath}
\usepackage{amssymb}
\usepackage{amsthm} %定理環境の拡張
\usepackage{ascmac}
\usepackage{bm}
\usepackage{cases}
\usepackage{comment} %非表示にするためのコメント
\usepackage{enumerate}
\usepackage{float} %画像をその場に表示.[h]の代わりに[H]
\usepackage{graphicx} % eps 形式の図版取り込みのため
\usepackage{mathrsfs}
\usepackage{url}
\usepackage[dvipdfmx]{hyperref}
%----------------------------------------------------------

%----------------------------------------------------------
%命題関係の定義
\theoremstyle{definition}
\newtheorem{definition}{定義}[section]
\newtheorem{theorem}{定理}[section]
\newtheorem{proposition}[theorem]{命題}
\newtheorem{lemma}[theorem]{補題}
\newtheorem{col}[theorem]{系}
\newtheorem{example}{例}[section]
\newtheorem{remark}{注意}[section]
%----------------------------------------------------------

%タイトル・著者===================================================
\title{レポート例}
\author{岡部 格明}
%=================================================================

%本文開始=========================================================
\begin{document}

\maketitle

%カウンタ--------------------------------------------------
%\setcounter{section}{0}
%\setcounter{subsection}{0}
%\setcounter{subsubsection}{0}
%\setcounter{theorem}{0}
%----------------------------------------------------------

\begin{proposition}
$\bm{A} \in \mathbb{R}^{p\times q}$, $\bm{B} \in \mathbb{R}^{p\times (p-q)}$, $[\bm{A},\bm{B}]$は正則行列,$\bm{A}^{\top}\bm{B} = \bm{O}$とする.\\
このとき,
\begin{align}
    \bm{A}(\bm{A}^{\top}\bm{A})^{-1}\bm{A}^{\top} + \bm{B}(\bm{B}^{\top}\bm{B})^{-1}\bm{B}^{\top} = I
\end{align}
が成り立つ.
\end{proposition}
% ---

\begin{proof}
$[\bm{A},\bm{B}]$の列空間への射影行列は
\begin{align*}
    [\bm{A},\bm{B}]([\bm{A},\bm{B}]^{\top}[\bm{A},\bm{B}])^{-1}[\bm{A},\bm{B}]^{\top}
    &=
    [\bm{A},\bm{B}]
    \begin{bmatrix}
        \bm{A}^{\top}\bm{A} & \bm{A}^{\top}\bm{B} \\
        \bm{B}^{\top}\bm{A} & \bm{B}^{\top}\bm{B} \\
    \end{bmatrix}^{-1}
    [\bm{A},\bm{B}]^{\top}
    \\
    &=
    [\bm{A},\bm{B}]
    \begin{bmatrix}
        \bm{A}^{\top}\bm{A} & \bm{O} \\
        \bm{O} & \bm{B}^{\top}\bm{B} \\
    \end{bmatrix}^{-1}
    [\bm{A},\bm{B}]^{\top}
    \\
    &=
    [\bm{A},\bm{B}]
    \begin{bmatrix}
        (\bm{A}^{\top}\bm{A})^{-1} & \bm{O} \\
        \bm{O} & (\bm{B}^{\top}\bm{B})^{-1} \\
    \end{bmatrix}
    [\bm{A},\bm{B}]^{\top}
    \\
    &=
    \begin{bmatrix}
        \bm{A}(\bm{A}^{\top}\bm{A})^{-1} ,&
        \bm{B}(\bm{B}^{\top}\bm{B})^{-1} \\
    \end{bmatrix}
    \begin{bmatrix}
        \bm{A}^{\top} \\
        \bm{B}^{\top} \\
    \end{bmatrix}
    \\
    &= \bm{A}(\bm{A}^{\top}\bm{A})^{-1}\bm{A}^{\top} + \bm{B}(\bm{B}^{\top}\bm{B})^{-1}\bm{B}^{\top}
\end{align*}
となる.

一方で,$\bm{A}$の列空間への射影$\bm{\bm{P}}_{\bm{A}}$を考えると,
\begin{align*}
    \bm{P}_{\bm{A}} = \bm{A}(\bm{A}^{\top}\bm{A})^{-1}\bm{A}^{\top}
\end{align*}
であり,$\bm{A}$の列空間と直交する空間への射影を$\bm{P}^{\bot}_{\bm{A}}$とすると
\begin{align*}
    \bm{P}_{\bm{A}} + \bm{P}^{\bot}_{\bm{A}} = I
\end{align*}
が成り立つ.
いま$\bm{A}^{\top}\bm{B}=\bm{O}$より,$\bm{B}$の列空間と$\bm{A}$の列空間が直交することから,
\begin{align*}
    \bm{P}_{\bm{A}} + \bm{P}_{\bm{B}} = I
\end{align*}
が成り立つ.

よって,
$\bm{A}(\bm{A}^{\top}\bm{A})^{-1}\bm{A}^{\top} + \bm{B}(\bm{B}^{\top}\bm{B})^{-1}\bm{B}^{\top} = I$が成り立つ.

\end{proof}


\end{document}
%本文ここまで=========================================================
