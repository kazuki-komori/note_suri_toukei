\synctex=1
\documentclass[dvipdfmx,10pt, a4j]{jarticle}
%----------------------------------------------------------
%パッケージ読み込み
\usepackage{amsmath}
\usepackage{amssymb}
\usepackage{amsthm} %定理環境の拡張
\usepackage{ascmac}
\usepackage{bm}
\usepackage{cases}
\usepackage{comment} %非表示にするためのコメント
\usepackage{enumerate}
\usepackage{float} %画像をその場に表示.[h]の代わりに[H]
\usepackage{graphicx} % eps 形式の図版取り込みのため
\usepackage{mathrsfs}
\usepackage{url}
\usepackage[dvipdfmx]{hyperref}
%----------------------------------------------------------

%----------------------------------------------------------
%命題関係の定義
\theoremstyle{definition}
\newtheorem{definition}{定義}[section]
\newtheorem{theorem}{定理}[section]
\newtheorem{proposition}[theorem]{命題}
\newtheorem{lemma}[theorem]{補題}
\newtheorem{col}[theorem]{系}
\newtheorem{example}{例}[section]
\newtheorem{remark}{注意}[section]
%----------------------------------------------------------

%タイトル・著者===================================================
\title{第3回 数理統計 レポート}
\author{小森 一輝}
%=================================================================

%本文開始=========================================================
\begin{document}

    \maketitle

%カウンタ--------------------------------------------------
    \setcounter{section}{2}
%\setcounter{subsection}{0}
%\setcounter{subsubsection}{0}
%\setcounter{theorem}{0}
%----------------------------------------------------------定理2.8
    \noindent
    \textbf{定理 2.8.} 完全劣加法性\\
    $A_i \in \mathcal{A}\, (i = 1, 2, \dots)$ ならば, 以下が成り立つ。\\
    \begin{align*}
        P(\bigcup_{i=1}^{\infty}{A_i}) \, \leq \, \sum_{i=1}^{\infty}P(A_i)
    \end{align*}
    この性質を \textbf{完全劣加法性} という。\\

%----------------------------------------------------------系2.9.
    \noindent
    \textbf{系 2.9.} 有限劣加法性\\
    $A_i \in \mathcal{A}\, (i = 1, 2, \dots, n)$ ならば, 以下が成り立つ。\\
    \begin{align*}
        P(\bigcup_{i=1}^{n}{A_i}) \, \leq \, \sum_{i=1}^{n}P(A_i)
    \end{align*}
    この不等式は \textbf{ブールの不等式} とよばれ、この性質を \textbf{有限劣加法性} という。\\

    %---------------以下補足
    \begin{itembox}[l]{補足}
        \begin{flushleft}
            $B_1 = A_1$\\
            $B_2 = A_2 \cap {A_1}^{c}$\\
            $A_1 \cup A_2 = B_1 \cup B_2$\\
            $B_i \subset A_i$
        \end{flushleft}
    \end{itembox}

    % ---------------以下証明
    \newpage
    \begin{proof}
        $A_i$ を互いに排反な事象 $B_i$ で表す。\\
        \begin{align*}
            & \begin{cases}
                  B_1 = A_1\\
                  B_i = A_i \, \cap \, (\, \bigcup_{k=1}^{i-1}{A_k} \, )^{c} \qquad  (i \geqq 2)
            \end{cases}
            とおく。\\
            & \begin{cases}
                  \bigcup_{i=1}^{\infty}{A_i} = \bigcup_{i=1}^{\infty}{B_i} ・・・※\\
                  B_1 \subset A_i (i = 1, 2, \dots)
            \end{cases}
            が成り立つ。\\
            ※\,より、\\
            P(\bigcup_{i=1}^{\infty}{A_i}) &= P(\bigcup_{i=1}^{\infty}{B_i})\\
            &= \sum_{i=1}^{\infty}{P(B_i)} \qquad (\because{B_i\, が互いに排反})\\
            したがって、B_i \in A_i であるから,\\
            \sum_{i=1}^{\infty}{P(B_i)} \leq \sum_{i=1}^{\infty}{P(A_i)}
        \end{align*}
    \end{proof}

%----------------------------------------------------------定理2.10
    \noindent
    \textbf{定理 2.10.} ボンフェローニの不等式\\
    $A_i \in \mathcal{A}\, (i = 1, 2, \dots, n)$ ならば, 以下が成り立つ。\\
    \begin{align*}
        P(\bigcap_{i=1}^{n}{A_i}) \, \leq \, 1 - \sum_{i=1}^{n}P({A_i}^c)
    \end{align*}
    この不等式を \textbf{ボンフェローニの不等式} という。\\
    % ---------------以下証明
    \begin{proof}
        \begin{align*}
            P(\bigcap_{i=1}^{n}{A_i}) &= P((\bigcup_{i=1}^{n}{{A_i}^c})^{c})\\
            &= 1 - P(\bigcup_{i=1}^{n}{{A_i}^c})\\
            &\geq 1 - \sum_{i=1}^{n}P(A_i) \qquad (\because{ブールの不等式})
        \end{align*}
    \end{proof}

%----------------------------------------------------------定理2.11
    \noindent
    \textbf{定理 2.11.} 条件付き確率\\
    確率空間 $(\Omega, \mathcal{A}, P)$ において,\,
    事象B $ \in \mathcal{A}$ (ただし, $P(B) > 0$) が与えられているとき, \, 以下で定義される実数関数$P(\, ・ \mid B)$ は確率測度となる。
    \begin{align*}
        P(A \mid B) = \frac{P(A \cap B)}{P(B)} \qquad (A \in \mathcal{A})
    \end{align*}
    このとき$P(A \mid B)$ を事象Bが起こったという条件のもとでの事象Aの条件付き確率とよぶ。\\
    \begin{enumerate}
        \item $0 \leq P(A \mid B) \leq 1$
        \item $P(\Omega \mid B) = 1$
        \item $A_i (i = 1, 2, \dots) \in \mathcal{A}$ を互いに排反とすると、\\
        $P\, \left(\bigcup_{i=1}^{\infty}{A_i \mid B}\right) = \sum_{i=1}^{\infty}{P\, (A_i \mid B)}$
    \end{enumerate}
    %---------------以下補足
    \begin{itembox}[l]{注}
        \begin{flushleft}
            $P\, (・ \mid B)$\, と\, P(・) の2つは異なる関数
        \end{flushleft}
    \end{itembox}\\

%----------------------------------------------------------定理2.15
    \noindent
    \textbf{定理 2.15.} A, B $\in \mathcal{A}, P(A), P(B) > 0$ について, \, 以下が成り立つ。\\
    \begin{enumerate}[i)]
        \item $P(A \cap B) = P(A)P(B \mid A) = P(B)P(A \mid B) \qquad (乗法定理)$\\
        % ---------------以下証明
        \begin{proof}
            \begin{align*}
                P(A)P(B \mid A) = P(A)\frac{P(A)}{P(B \cap A)} = P(A \cap B)\\
                P(B)P(A \mid B) = P(B)\frac{P(B)}{P(A \cap B)} = P(A \cap B)
            \end{align*}
        \end{proof}
        \item $P(A \mid B) = P(A) ならば, P(B \mid A) = P(B)$
        % ---------------以下証明
        \begin{proof}
            \begin{align*}
                P(B \mid A) = \frac{P(B \cap A)}{P(A)} = \frac{P(B \cap A)}{P(A \mid B)} = P(B)
            \end{align*}
        \end{proof}
        \item $P(A \cap B) \leq P(B \mid A), P(B \cap A) \leq P(A \mid B)$
        % ---------------以下証明
        \begin{proof}
            \begin{align*}
                &P(B \mid A) - P(A \mid B)\\
                &= \frac{P(B \cap A)}{P(A)} - P(A \cap B)\\
                &= (A \cap B)(\frac{1}{P(A)} - 1)\\
                &= (A \cap B)(\frac{1 - P(A)}{P(A)}) \geq 0 \qquad (\because{P(A) \leq 1})
            \end{align*}
        \end{proof}
    \end{enumerate}

%----------------------------------------------------------定理2.16
    \noindent
    \textbf{定理 2.16.} $A, A_i \in \mathcal{A}\, (i = 1, 2, \dots, n), P(A) > 0 について, \bigcup_{i=1}^{n}{A_i} = \Omega, A_i \cap A_j = \phi\, (i \neq j)ならば, 以下が成り立つ。$\\
    \begin{align*}
        \sum_{i=1}^{n}{P(A_i \mid A)} = 1
    \end{align*}
    % ---------------以下証明
    \begin{proof}
        \begin{align*}
            \sum P(A_i \mid A) &= \sum \frac{P(A_i \cap A)}{P(A)}\\
            &= \frac{1}{P(A)} \sum P(A_i \cap A)\\
            ここで、A_i \cap A = B_i とする。\\
            &= \frac{1}{P(A)} P(\bigcup_{i=1}^{n}{B_i})\\
            &= \frac{1}{P(A)} P(\bigcup_{i=1}^{n}{(A_i \cap A)})\\
            &= \frac{1}{P(A)} P(\bigcup_{i=1}^{n}{(A_i)} \cap A)\\
            &= \frac{1}{P(A)} P(\Omega \cap A)\\
            &= \frac{P(A)}{P(A)} = 1\\
        \end{align*}
    \end{proof}
    %---------------以下補足
    \begin{itembox}[l]{補足}
        \begin{align*}
            \bigcup_{i=1}^{n}{A_i} = \Omega,\, A_i \cap A_j = \phi \qquad (i \neq j)のとき, A_i は \Omega を分割したものである。\\
        \end{align*}
        また、$\bigcup_{i=1}^{n}{A_i \cap A}$は以下のように変形できる。\\
        \begin{align*}
            \bigcup_{i=1}^{n}{A_i \cap A} = (\bigcup_{i=1}^{n}{A_i}) \cap A = \Omega \cap A = A
        \end{align*}
    \end{itembox}\\

%----------------------------------------------------------例2.7
    \newpage
    \noindent
    \textbf{例 2.7.} $A, B, C \in \mathcal{A}$ について, 以下が成り立つ。\\
    \begin{align*}
        P(A \cap B \cap C) = P(A)P(B \mid A)P(C \mid A \cap B)
    \end{align*}

    % ---------------以下証明
    \begin{proof}
        \begin{align*}
            P(A)P(B \mid A)P(C \mid A \cap B) &= P(A) \frac{P(A)}{P(A \cap B)} \frac{P(C \cap B \cap A)}{P(A \cap B)}\\
            &= P(A \cap B \cap C)
        \end{align*}
    \end{proof}
    %---------------以下補足
    \begin{itembox}[l]{注}
        積集合は条件付き確率の積であって、確率の積ではない。\\
        \begin{align*}
            P(A \cap B) &= P(A)P(B \mid A)\\
            &\neq P(A)・P(B)\\
        \end{align*}
    \end{itembox}\\

%----------------------------------------------------------定理2.17
    \noindent
    \textbf{定理 2.17.} 一般乗法定理\\
    $A_i \in \mathcal{A} (i = 1,2, \dots, n)m P\left(\bigcap_{i=1}^{n-1}{A_i}\right) > 0 ならば, 以下が成り立つ。$\\
    \begin{align*}
        P(\bigcap_{i=1}^{n}{A_i}) &= P(A_1)P(A_2 \mid A_1)P(A_3 \mid A_1 \cap A_2)\\
        & × \cdots × P(A_i \mid \bigcap_{i=1}^{l-1}{A_i}) × \cdots × P(A_n \mid \bigcap_{i=1}^{n-1}{A_i})\\
    \end{align*}
    この性質は\textbf{一般乗法定理}とよばれる。(帰納法で証明可能)\\

%----------------------------------------------------------例2.8
    \noindent
    \textbf{例 2.8.} ポリアの壺\\
    1つの壺に白玉w個, 黒玉b個入っている。この壺から無作為に1個の玉を取り出す。この玉を元の壺に返し,その際に,取り出された多摩と同じ色の玉を$c(>0)$個壺に入れる。
    このような試行をn回行う。i番目に取り出された玉の色が白あるいは黒である事象をそれぞれ$W_i, B_i(i=1,2,\dots,n)$と記す。このとき,\\
    \begin{align*}
        P(W_i) = \frac{w}{w+b},\qquad P(B_i) = \frac{b}{w+b} \qquad(i=1,2,\dots,n)\\
    \end{align*}
    が成り立つ。この例を \textbf{ポリアの壺}とよばれる。\\
    % ---------------以下証明
    \begin{proof}
        i = 1のとき,以下のように表される。\\
        \begin{align*}
            P(W_1) = \frac{w}{w+b}, \qquad P(B_1) = \frac{b}{w+b}
        \end{align*}
        i = 2のときを考える。\\
        \begin{align*}
            P(W_2 \mid W_1) = \frac{w + c}{w + b + c} \, \neq \, P(W_2) \qquad P(W_2 \mid B_1) = \frac{w}{w + b + c}\\
            P(B_2 \mid B_1) = \frac{b + c}{w + b + c} \qquad P(B_2 \mid B_1) = \frac{b}{w + b + c}\\
        \end{align*}

        \begin{align*}
            P(W_1 \cap W_2) = P(W_1)P(W_2 \mid W_1) = \frac{w}{w+b} \cdot \frac{w+c}{w+b+c}\\
            P(B_1 \cap B_2) = P(B_1)P(B_2 \mid B_1) = \frac{b}{w+b} \cdot \frac{b+c}{w+b+c}\\
            P(W_1 \cap B_2) = P(W_1)P(B_2 \mid W_1) = \frac{w}{w+b} \cdot \frac{b}{w+b+c}\\
            P(B_1 \cap W_2) = P(B_1)P(W_2 \mid B_1) = \frac{b}{w+b} \cdot \frac{w}{w+b+c}\\
        \end{align*}

        \begin{align*}
            W_2 = W_2 \cap (W_1 \cup B_1) = (W_2 \cap W_1) \cap (W_2 \cup B_1) より、
        \end{align*}
        \begin{align*}
            P(W_2) &= P(W_1 \cap W_2) + P(W_1 \cap W_2)\\
            &= \frac{w}{w+b} \cdot \frac{w+c}{w+b+c} + \frac{b}{w+b} \cdot \frac{b}{w+b+c}\\
            &= \frac{w(w+c+b)}{(w+b)(w+b+c)} = \frac{w}{w+b}\\
        \end{align*}
    \end{proof}

    %----------------------------------------------------------定理2.18
    \noindent
    \textbf{定理 2.18.} 全確率の定理\\
    $A_i, B \in \mathcal{A} (i = 1, 2, \dots, n)に対して, \bigcup_{i=1}^{n}{A_i} = \Omega, A_i \cap A_j \, (i \neq j) が成り立ち,P(A_i) > 0 ならば ,以下が成り立つ。$
    \begin{align*}
        P(B) = \sum_{i=1}^{n}{P(A_i)P(B \mid A_i)}\\
    \end{align*}
    この性質は, \textbf{全確率の定理}とよばれる。\\

    % ---------------以下証明
    \begin{proof}
        \begin{align*}
            P(B) &= P(B \cap \Omega)\\
            &= P(B \cap (\bigcup_{i=1}^{n}{A_i}))\\
            &= P(\bigcup_{i=1}^{n}{(B \cap A_i)}))\\
            &= \sum_{i=1}^{n}{P(B \cap A_i)}\\
            &= \sum_{i=1}^{n}{P(A_i)P(B \mid A_i)}\\
        \end{align*}
    \end{proof}

    %----------------------------------------------------------定理2.19
    \noindent
    \textbf{定理 2.19.} ベイズの定理\\
    $A_i, B \in \mathcal{A} (i = 1, 2, \dots, n)に対して, \bigcup_{i=1}^{n}{A_i} = \Omega, A_i \cap A_j \, (i \neq j) が成り立ち,P(A_i), P(B) > 0 ならば ,以下が成り立つ。$
    \begin{align*}
        P(A_j \mid B) = \frac{P(A_j)P(B \mid A_j)}{\sum_{i=1}^{n}{P(A_i)P(B \mid A_i)}} \qquad (j = 1, 2, \dots, n)\\
    \end{align*}
    この性質は, \textbf{ベイズの定理}とよばれる。\\
    %---------------以下補足
    \begin{itembox}[l]{補足}
        事前確率:$P(A_i) \cdots 事象Bが起きる前$\\
        事後確率:$P(A_i \mid B) \cdots 事象Bが起きた後$\\
    \end{itembox}\\


%----------------------------------------------------------例2.9
    \noindent
    \textbf{例 2.9.} モンティホール問題\\
    存在する3つのドアの名前ををA, B, C, ドアが当たりである確率をP(X), P(Y)をホストが開けるドアの確率とおく。\\
    ここで, プレイヤーがドアAを常に最初に選択するという仮定のもと,プレイヤーが当たりのドアを開ける条件付き確率を考える。\\
    プレイヤーが最初の選択を変えないとき\,($P(Y = A \mid X = B) = \frac{1}{2}$),以下のようになる。
    \begin{align*}
        P(X = B \mid Y= A) = \frac{P(Y = A \mid X = B)}{P(Y = A)}P(X = B) = \frac{1}{3}
    \end{align*}
    プレイヤーが最初の選択を変えるとき\, ($P(Y = A \mid X = C) = 1$),以下のようになる。
    \begin{align*}
        P(X = C \mid Y= A) = \frac{P(Y = A \mid X = C)}{P(Y = A)}P(X = C) = \frac{2}{3}
    \end{align*}
    したがって、プレイヤーは最初の選択を変えたときのほうが当たりやすいと言える。

\end{document}
%本文ここまで=========================================================
