\synctex=1
\documentclass[dvipdfmx,10pt, a4j]{jarticle}
%----------------------------------------------------------
%パッケージ読み込み
\usepackage{amsmath}
\usepackage{amssymb}
\usepackage{amsthm} %定理環境の拡張
\usepackage{ascmac}
\usepackage{bm}
\usepackage{cases}
\usepackage{comment} %非表示にするためのコメント
\usepackage{enumerate}
\usepackage{float} %画像をその場に表示.[h]の代わりに[H]
\usepackage{graphicx} % eps 形式の図版取り込みのため
\usepackage{mathrsfs}
\usepackage{url}
\usepackage[dvipdfmx]{hyperref}
%----------------------------------------------------------
%自作コマンド
\newcommand{\indepe}{\mathop{\perp\!\!\!\perp}}

%----------------------------------------------------------
%命題関係の定義
\theoremstyle{definition}
\newtheorem{definition}{定義}[section]
\newtheorem{theorem}{定理}[section]
\newtheorem{proposition}[theorem]{命題}
\newtheorem{lemma}[theorem]{補題}
\newtheorem{col}[theorem]{系}
\newtheorem{example}{例}[section]
\newtheorem{remark}{注意}[section]
%----------------------------------------------------------

%タイトル・著者===================================================
\title{第4回 数理統計 レポート}
\author{小森 一輝}
%=================================================================

%本文開始=========================================================
\begin{document}

    \maketitle

%カウンタ--------------------------------------------------
    \setcounter{section}{2}
%\setcounter{subsection}{0}
%\setcounter{subsubsection}{0}
%\setcounter{theorem}{0}
%----------------------------------------------------------定理2.12
    \noindent
    \textbf{定理 2.12.} 事象の独立\\
    $確率空間(\Omega, \mathcal{A}, P) において, A, B \in \mathcal{A}が,$\\
    \begin{align*}
        P(A \cap B) = P(A)P(B)
    \end{align*}
    を満たすとき, 事象AとBは \textbf{独立} であるといい、$A \indepe B$ と記す。\\
    また、一般に $P(A \cap B) = P(A)P(B \mid A)$が成りたち、事象A,Bが互いに独立の場合特に、$P(B \mid A) = P(B)$が成り立つ。\\

%----------------------------------------------------------定理2.20
    \noindent
    \textbf{定理 2.20.} $A, B \in \mathcal{A} について,以下が成り立つ。$\\
    \begin{enumerate}[i)]
        \item $A \indepe B, A \indepe B^{c}, A^{c} \indepe B, A^{c} \indepe B^{c} は動値の命題である。$
        % ---------------以下証明
        \begin{proof}
            \begin{align*}
                A \indepe B => A \indepe B^{c} を示す。\\
                A = (A \cap B) \cup (A \cap B^{c}), \qquad (A \cap B) \cap (A \cap B^{c}) = \phi
            \end{align*}
            \begin{align*}
                p(A \cap B^{c}) &= P(A) - P(A \cap B)\\
                &= P(A) - P(A) \cdot P(B) \qquad (\because{A \indepe B}) \\
                &= P(A)(1-P(B))\\
                &= P(A)P(B^{c})\\
            \end{align*}
        \end{proof}
        
        \newpage
        \item $任意の A \in \mathcal{A} に対して, A \indepe \Omega$
        % ---------------以下証明
        \begin{proof}
            \begin{align*}
                A \subset \Omega, \, P(\Omega) = 1\\
                A \subset \Omega \, より, \, A \cap \Omega = A
            \end{align*}
            したがって,\\
            \begin{align*}
                P(A \cap \Omega) = P(A) = P(A) \cdot P(\Omega) \qquad (\because p(\Omega) = 1)\\
            \end{align*}
        \end{proof}
        \item $N \in \mathcal{A} が P(N) = 0 を満たすとき, 任意のA \in \mathcal{A} に対して, A \indepe N$
        % ---------------以下証明
        \begin{proof}
            \begin{align*}
                N \in \mathcal{A}, \, P(N) = 0\\
                A \cap N \subset N より,
            \end{align*}
            \begin{align*}
                P(A \cap N) \leq P(N) = 0\\
                P(A) \cdot P(N) = 0\\
            \end{align*}
        \end{proof}
        \item $P(B) > 0のとき, A \indepe Bであるための必要十分条件は, \, P(A \mid B) = P(A) である。$
        % ---------------以下証明
        \begin{proof}
            \begin{align*}
                P(B) > 0のとき, A \indepe B \leftrightarrow p(A \mid B) = P(A)\\
            \end{align*}
            $\rightarrow (必要条件)$\\
            $A \indepe B とする$
            \begin{align*}
                P(A \mid B) = \frac{P(A \cap B)}{P(B)} = \frac{P(A) \cdot P(B)}{P(B)} = P(A)\\
            \end{align*}
            $\leftarrow (十分条件)$\\
            $P(A \mid B) = P(A) とする$
            \begin{align*}
                P(A \mid B) = \frac{P(A \cap B)}{P(B)} = P(A)\\
            \end{align*}
        \end{proof}
    \end{enumerate}

%----------------------------------------------------------定理2.13
\noindent
\textbf{定理 2.13.} 事象族の独立\\
$確率空間 (\Omega, \mathcal{A}, P)における事象列, \, \{A_i \mid i = 1,2, \dots ,n\}について, 任意の1 \leq m \leq n, 1 \leq i_1 \leq i_2 < \cdots < i_m \leq n に対して,$\\
\begin{align*}
    P(\bigcap_{j=1}^{m}{A_j}) = \prod_{j=1}^{m}{P(A_{ij})}\\
\end{align*}
が成り立つとき, 事象例 $\{A_i \mid i = 1,2, \dots ,n \}$ は互いに独立であると呼ばれる。\\
さらに,$確率空間 (\Omega, \mathcal{A}, P)$における事象族 $\{ A_{\lambda} \mid \lambda \in A \}について,その任意有限個の要素からなる族が互いに独立のとき,事象族$
$\{ A_{\lambda} \mid \lambda \in A \}$ は互いに独立であると呼ばれる。\\

\textbf{例} $n = 3 \rightarrow m = 2,3$\\
m = 3\\
\begin{enumerate}[1)]
    \item $P(A_1 \cap A_2 \cap A_3) = P(A_1)P(A_2)P(A_3)$\\
    \item $P(A_1 \cap A_2) = P(A_1)P(A_2)$\\
    \item $P(A_2 \cap A_3) = P(A_2)P(A_3)$\\
    \item $P(A_3 \cap A_1) = P(A_3)P(A_1)$\\
\end{enumerate}
1 $\to$ 4 すべて成り立つとき,事象列$A_1, A_2, A_3$は独立である。\\

\newpage
%----------------------------------------------------------例2.10
\noindent
\textbf{例 2.10.} $A, B, C が互いに独立ならば, A^{c}, B^{c} ,C^{c} も互いに独立となる。すなわち,$\\
\begin{proof}
    $A^{c} \indepe B^{c} \indepe C^{c}$を示す.\\
    \begin{align*}
        P((A \cup B) \cap C) &= P((A \cap C) \cup (B \cap C))\\
        &= P(A \cap C) + P(B \cap C) - P(A \cap B \cap C)\\
        &= P(A) \cdot P(C) + P(B) \cdot P(C) - P(A) \cdot P(B) \cdot P(C)\\
        &= (P(A) + P(B) - P(A)P(B))P(C) \qquad (\because それぞれの事象は独立)\\
        &= (P(A) + P(B) - P(A \cap B))P(C)\\
        &= P(A \cup B) \cdot P(C)\\
        &= (A \cup B) \indepe C \\
        &= (A \cup B) \indepe C^{c} \qquad (\because 定理 2.20-1)\\
        &= (A^{c} \cap B^{c}) \indepe C^{c} \qquad (\because ドモルガン)\\
        &= A^{c} \indepe B^{c} \indepe C^{c}\\
    \end{align*}
\end{proof}

%----------------------------------------------------------定理2.21
\noindent
\textbf{定理 2.21.} $確率空間 (\Omega, \mathcal{A}, P)における事象列 \{ A_i \mid i = 1, 2, \dots ,n \}が互いに独立のとき以下が成り立つ。$\\
\begin{align*}
    P(\bigcup_{i=1}^{n}{A_i}) = 1 - \prod_{i = 1}^{n}{P(A_i^{c})}\\
\end{align*}
\begin{proof}
    $(\{A_i\}が独立 \rightarrow \{A^{c}_i\}が独立 ※要証明)$\\
    \begin{align*}
        P(\bigcup_{i=1}^{n}{A_i}) &= 1 - P((\bigcup_{i=1}^{n}{A_i})^{c})\\
        &= 1 - P(\bigcap_{i=1}^{n}{A^{c}_i}) \qquad (\because ドモルガン)\\
        &= 1 - \prod_{i=1}^{n}{P(A^{c}_i)}\\
    \end{align*}
\end{proof}

\newpage
%----------------------------------------------------------定義3.1
\noindent
\textbf{定義 3.1.} 確率変数\\
$確率空間 (\Omega, \mathcal{A}, P)において, 写像X: \Omega \rightarrow \mathbb{R}が任意の集合B \in \mathcal{B}について,$\\
\begin{align*}
    X^{-1}(B) = \{ \omega \mid X(\omega) \in B\} \in \mathcal{A}\\
\end{align*}
を満たすならば, Xは\textbf{確率変数}とよばれる。\\
$B \in \mathcal{B}を区間全体の集合\textbf{ボレル集合}とよぶ。$\\
$X^{-1}(B) \in \mathcal{A}のとき, XによるBの逆像と表現する。$

\end{document}
%本文ここまで=========================================================

