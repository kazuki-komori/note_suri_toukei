\synctex=1
\documentclass[dvipdfmx,10pt, a4j]{jarticle}
%----------------------------------------------------------
%パッケージ読み込み
\usepackage{amsmath}
\usepackage{amssymb}
\usepackage{amsthm} %定理環境の拡張
\usepackage{ascmac}
\usepackage{bm}
\usepackage{cases}
\usepackage{comment} %非表示にするためのコメント
\usepackage{enumerate}
\usepackage{float} %画像をその場に表示.[h]の代わりに[H]
\usepackage{graphicx} % eps 形式の図版取り込みのため
\usepackage{mathrsfs}
\usepackage{url}
\usepackage[dvipdfmx]{hyperref}
%----------------------------------------------------------
%自作コマンド
\newcommand{\indepe}{\mathop{\perp\!\!\!\perp}}

%----------------------------------------------------------
%命題関係の定義
\theoremstyle{definition}
\newtheorem{definition}{定義}[section]
\newtheorem{theorem}{定理}[section]
\newtheorem{proposition}[theorem]{命題}
\newtheorem{lemma}[theorem]{補題}
\newtheorem{col}[theorem]{系}
\newtheorem{example}{例}[section]
\newtheorem{remark}{注意}[section]
%----------------------------------------------------------

%タイトル・著者===================================================
\title{第4回 数理統計 レポート}
\author{小森 一輝}
%=================================================================

%本文開始=========================================================
\begin{document}

    \maketitle

%カウンタ--------------------------------------------------
    \setcounter{section}{2}
%\setcounter{subsection}{0}
%\setcounter{subsubsection}{0}
%\setcounter{theorem}{0}
%----------------------------------------------------------定理2.8
    \noindent
    \textbf{定理 2.12.} 事象の独立\\
    $確率空間(\Omega, \mathcal{A}, P) において, A, B \in \mathcal{A}が,$\\
    \begin{align*}
        P(A \cap B) = P(A)P(B)
    \end{align*}
    を満たすとき, 事象AとBは \textbf{独立} であるといい、$A \indepe B$ と記す。\\
    また、一般に $P(A \cap B) = P(A)P(B \mid A)$が成りたち、事象A,Bが互いに独立の場合特に、$P(B \mid A) = P(B)$が成り立つ。\\

%----------------------------------------------------------定理2.8
    \noindent
    \textbf{定理 2.20.} $A, B \in \mathcal{A} について,以下が成り立つ。$\\
    \begin{enumerate}[i)]
        \item $A \indepe B, A \indepe B^{c}, A^{c} \indepe B, A^{c} \indepe B^{c} は動値の命題である。$
        \begin{proof}
            \begin{align*}
                A \indepe B => A \indepe B^{c} を示す。\\
                A = (A \cap B) \cup (A \cap B^{c}), \qquad (A \cap B) \cap (A \cap B^{c}) = \phi\\
                p(A \cap B^{c}) &= P(A) - P(A \cap B)\\
                &= P(A) - P(A) \cdot P(B)\\
            \end{align*}
        \end{proof}
    \end{enumerate}

\end{document}
%本文ここまで=========================================================
