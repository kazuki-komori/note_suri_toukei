\synctex=1
\documentclass[dvipdfmx,10pt, a4j]{jarticle}
%----------------------------------------------------------
%パッケージ読み込み
\usepackage{amsmath}
\usepackage{amssymb}
\usepackage{amsthm} %定理環境の拡張
\usepackage{ascmac}
\usepackage{bm}
\usepackage{cases}
\usepackage{comment} %非表示にするためのコメント
\usepackage{enumerate}
\usepackage{float} %画像をその場に表示.[h]の代わりに[H]
\usepackage{graphicx} % eps 形式の図版取り込みのため
\usepackage{mathrsfs}
\usepackage{url}
\usepackage[dvipdfmx]{hyperref}
%----------------------------------------------------------

%----------------------------------------------------------
%命題関係の定義
\theoremstyle{definition}
\newtheorem{definition}{定義}[section]
\newtheorem{theorem}{定理}[section]
\newtheorem{proposition}[theorem]{命題}
\newtheorem{lemma}[theorem]{補題}
\newtheorem{col}[theorem]{系}
\newtheorem{example}{例}[section]
\newtheorem{remark}{注意}[section]
%----------------------------------------------------------

%タイトル・著者===================================================
\title{第3回 データベースシステム 課題}
\author{1108190116 \, 小森 一輝}
%=================================================================

%本文開始=========================================================
\begin{document}

    \maketitle

%カウンタ--------------------------------------------------
    \setcounter{section}{2}
%\setcounter{subsection}{0}
%\setcounter{subsubsection}{0}
%\setcounter{theorem}{0}
%----------------------------------------------------------問1
    \noindent
    \textbf{問1} $(\forall_x)(f(x)OR(\exists_y)(f(y) AND f(x)) IMP f(x))$ \\
    \begin{align*}
        (与式) &= (\forall_x)(f(x)\, OR\, NOT\, (\exists_y)(f(y)\, AND\, f(x))\, OR\, f(x))\\
        &= (\forall_x)(f(x)\,OR\,NOT\,(\exists_y)f(y)\,AND\,NOT\,f(x)\,OR\,f(x))\\
        &= (\forall_x)(F\, OR\, NOT\,(\exists_y)f(y)\,OR\,f(x))\\
        &= NOT\, (\exists_y)f(y)\, OR\, (\forall_x)\, f(x)\\
        &= NOT\, (\exists_x)f(x)\, OR\, (\forall_x)\, f(x)\\
        &= T\\
    \end{align*}

%----------------------------------------------------------問2
    \noindent
    \textbf{問2} $(\forall_x)(g(x)\, IMP \, (\forall_y)(g(y)\, IMP\, (\exists_z)(g(x)\, AND \, g(z)\, IMP\, g(y))))$ \\
    \begin{align*}
        (与式) &= (\forall_x)(g(x)\, IMP \, (\forall_y)(g(y)\, IMP\, (\exists_z)(g(x)\, AND\, NOT\, g(z)\, OR\, g(y))))\\
        &= (\forall_x)(NOT\, g(x)\, OR \, (\forall_y)(NOT\, g(y)\, OR\, (\exists_z)(g(x)\, AND\, NOT\, g(z)\, OR\, g(y))))\\
        &= (\forall_x)(NOT\, g(x)\, OR \, (\forall_y)(NOT\, g(y)\, OR\, g(x)\, AND\, NOT\, (\exists_z)\, g(z)\, OR\, g(y)))\\
        &= (\forall_x)(NOT\, g(x)\, OR\, T\, OR\, g(x)\, AND\, NOT\, (\exists_z)\, g(z))\\
        &= (T\, AND\, NOT\, (\exists_z)\, g(z))\\
        &= NOT\, (\exists_z)\, g(z)\\
        &= NOT\, (\exists_x)\, g(x)\\
    \end{align*}

%----------------------------------------------------------問3
    \noindent
    \textbf{問3} $(\exists_x)(f(x)\, AND \, (\forall_y)(f(x)\, IMP\, f(y))\, OR\, (\exists_z)(NOT\, f(z)\, OR\, NOT\, f(x)))$ \\
    \begin{align*}
        (与式) &= (\exists_x)(f(x)\, AND \,NOT\, f(x)\, OR\, (\forall_y)\, f(y)\, OR\, (\exists_z)(NOT\, f(z)\, OR\, NOT\, f(x)))\\
        &= (\exists_x)(F\, OR\, (\forall_y)\, f(y)\, OR\, NOT\, (\exists_z)f(z)\, OR\, NOT\, f(x))\\
        &= (\forall_y)\, f(y)\, OR\, NOT\, (\exists_z)f(z)\, OR\, NOT\, (\exists_x)f(x))\\
        &= (\forall_x)\, f(x)\, OR\, NOT\, (\exists_x)f(x)\\
        &= T\\
    \end{align*}
\end{document}
%本文ここまで=========================================================
