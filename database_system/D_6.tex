\synctex=1
\documentclass[dvipdfmx,10pt, a4j]{jarticle}
%----------------------------------------------------------
%パッケージ読み込み
\usepackage{amsmath}
\usepackage{amssymb}
\usepackage{amsthm} %定理環境の拡張
\usepackage{ascmac}
\usepackage{bm}
\usepackage{cases}
\usepackage{comment} %非表示にするためのコメント
\usepackage{enumerate}
\usepackage{float} %画像をその場に表示.[h]の代わりに[H]
\usepackage{graphicx} % eps 形式の図版取り込みのため
\usepackage{mathrsfs}
\usepackage{url}
\usepackage[dvipdfmx]{hyperref}
%----------------------------------------------------------

%----------------------------------------------------------
%命題関係の定義
\theoremstyle{definition}
\newtheorem{definition}{定義}[section]
\newtheorem{theorem}{定理}[section]
\newtheorem{proposition}[theorem]{命題}
\newtheorem{lemma}[theorem]{補題}
\newtheorem{col}[theorem]{系}
\newtheorem{example}{例}[section]
\newtheorem{remark}{注意}[section]
%----------------------------------------------------------

%タイトル・著者===================================================
\title{第6回 データベースシステム 課題}
\author{1108190116 \, 小森 一輝}
%=================================================================

%本文開始=========================================================
\begin{document}

\maketitle

%カウンタ--------------------------------------------------
\setcounter{section}{2}
%\setcounter{subsection}{0}
%\setcounter{subsubsection}{0}
%\setcounter{theorem}{0}

%----------------------------------------------------------定理2.8
\noindent
\textbf{課題} \\
\begin{align*}
    &主キー \dots \textbf{科目番号, 学籍番号}\\
    &外部キー \dots \textbf{履修:\; 学籍番号, 履修:\; 科目番号}\\
\end{align*}
\begin{align*}
    &与えられたデータより, それぞれの集合は以下のように表現することができる.\\
    &学籍番号: \{20001, 20002, 20003\}\\
    &学生名: \{杉下右京, 冠城亘, 小出茉梨\}\\
    &科目番号: \{CIS01, CIS02\}\\
    &科目名: \{情報アクセス技術, データベースシステム\}\\
    &得点: \{60, 70, 80, 90\}\\
\end{align*}
\begin{align*}
    &与えられたデータベースを以下のような形式で表現する.\\
    &\{学籍番号, 学生名, 科目番号, 科目名, 得点\}\\
    &\{20001, 杉下右京, CIS01, 情報アクセス技術, 60\}\\
    &\{20002, 冠城亘, CIS01, 情報アクセス技術, 90\}\\
    &\{20002, 冠城亘, CIS02, データベースシステム, 70\}\\
    &\{20003, 小出茉梨, CIS02, データベースシステム, 80\}\\
    &よって, 5項目の各集合, 計144通りの組み合わせの中に上記の項目が存在することがわかるから,\\
    &このデータベースは各集合の直積集合の部分集合であることがわかる.\\
\end{align*}

\end{document}
%本文ここまで=========================================================