\synctex=1
\documentclass[dvipdfmx,10pt, a4j]{jarticle}
%----------------------------------------------------------
%パッケージ読み込み
\usepackage{amsmath}
\usepackage{amssymb}
\usepackage{amsthm} %定理環境の拡張
\usepackage{ascmac}
\usepackage{bm}
\usepackage{cases}
\usepackage{comment} %非表示にするためのコメント
\usepackage{enumerate}
\usepackage{float} %画像をその場に表示.[h]の代わりに[H]
\usepackage{graphicx} % eps 形式の図版取り込みのため
\usepackage{mathrsfs}
\usepackage{url}
\usepackage[dvipdfmx]{hyperref}
%----------------------------------------------------------

%----------------------------------------------------------
%命題関係の定義
\theoremstyle{definition}
\newtheorem{definition}{定義}[section]
\newtheorem{theorem}{定理}[section]
\newtheorem{proposition}[theorem]{命題}
\newtheorem{lemma}[theorem]{補題}
\newtheorem{col}[theorem]{系}
\newtheorem{example}{例}[section]
\newtheorem{remark}{注意}[section]
%----------------------------------------------------------

%タイトル・著者===================================================
\title{第7回 データベースシステム 課題}
\author{1108190116 \, 小森 一輝}
%=================================================================

%本文開始=========================================================
\begin{document}

\maketitle

%カウンタ--------------------------------------------------
\setcounter{section}{2}
%\setcounter{subsection}{0}
%\setcounter{subsubsection}{0}
%\setcounter{theorem}{0}

%----------------------------------------------------------定理2.8
\noindent
    \textbf{プロセス1} $\rho(受注1, 受注)$ \\
    \begin{align*}
        &受注1\\
        &\begin{tabular}{|c|c|c|c|}
            \hline
            受注番号 & 受注数\\
            \hline
            \hline
            S010 & 300\\
            \hline
            S025 & 240\\
            \hline
            S045 & 280\\
            \hline
        \end{tabular}\\
    \end{align*}
    \textbf{プロセス2} $\rho(受注2, ^{}_{\rho 受注.受注数 \leqq 受注1.受注数}受注 × 受注1)$ \\
    \begin{align*}
        &受注2\\
        &\begin{tabular}{|c|c|c|c|}
            \hline
            受注.受注番号 & 受注.受注数 & 受注1.受注番号 & 受注1.受注数\\
            \hline
            \hline
            S010 & 300 & S010 & 300\\
            \hline
            S025 & 240 & S010 & 300\\
            \hline
            S045 & 280 & S010 & 300\\
            \hline
            S025 & 240 & S025 & 240\\
            \hline
            S025 & 240 & S045 & 280\\
            \hline
            S045 & 280 & S045 & 280\\
            \hline
        \end{tabular}\\
    \end{align*}
    \newpage
    \textbf{プロセス3} $^{}_{\pi 受注.受注番号}(受注2 \div 受注1)$ \\
    \begin{align*}
        \begin{tabular}{|c|c|c|c|}
            \hline
            受注.受注番号\\
            \hline
            \hline
            S025\\
            \hline
        \end{tabular}\\
    \end{align*}

\end{document}
%本文ここまで=========================================================