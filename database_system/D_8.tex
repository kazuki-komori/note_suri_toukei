\synctex=1
\documentclass[dvipdfmx,10pt, a4j]{jarticle}
%----------------------------------------------------------
%パッケージ読み込み
\usepackage{amsmath}
\usepackage{amssymb}
\usepackage{amsthm} %定理環境の拡張
\usepackage{ascmac}
\usepackage{bm}
\usepackage{cases}
\usepackage{comment} %非表示にするためのコメント
\usepackage{enumerate}
\usepackage{float} %画像をその場に表示.[h]の代わりに[H]
\usepackage{graphicx} % eps 形式の図版取り込みのため
\usepackage{mathrsfs}
\usepackage{url}
\usepackage[dvipdfmx]{hyperref}
\usepackage{color}
%----------------------------------------------------------

%----------------------------------------------------------
%命題関係の定義
\theoremstyle{definition}
\newtheorem{definition}{定義}[section]
\newtheorem{theorem}{定理}[section]
\newtheorem{proposition}[theorem]{命題}
\newtheorem{lemma}[theorem]{補題}
\newtheorem{col}[theorem]{系}
\newtheorem{example}{例}[section]
\newtheorem{remark}{注意}[section]
%----------------------------------------------------------

%タイトル・著者===================================================
\title{第8回 データベースシステム 課題}
\author{1108190116\; 小森 一輝}
%=================================================================

%本文開始=========================================================
\begin{document}

\maketitle

%カウンタ--------------------------------------------------
\setcounter{section}{0}
\setcounter{subsection}{0}
%\setcounter{subsubsection}{0}
%\setcounter{theorem}{0}

%----------------------------------------------------------1
\noindent
\section{社員表から,部門番号が0120である社員の社員番号と社員名,部門番号
  を出力せよ}

\begin{flushleft}
    SELECT\; 社員番号, 社員名, 部門番号\\
    FROM 社員\\
    WHERE 社員番号=0120\\
\end{flushleft}
%----------------------------------------------------------2
\noindent
\section{社員表から,部門番号が0110または0130である社員の社員番号と社員
  名,部門番号を二通りの方法で出力せよ}

\begin{flushleft}1通り目\\
    SELECT 社員番号, 社員名, 部門番号\\
    FROM 社員\\
    WHERE 部門番号=0110 OR 部門番号=0130\\
\end{flushleft}

\begin{flushleft}2通り目\\
    SELECT 社員番号, 社員名, 部門番号\\
    FROM 社員\\
    WHERE 部門番号 IN \\
    (SELECT 部門番号FROM 部門 WHERE 部門番号=0110 OR 部門番号=0130)\\
\end{flushleft}

%----------------------------------------------------------3
\noindent
\section{社員表から,社員番号と社員名,給料を,給料の高い社員から順に出力せ
  よ}

\begin{flushleft}
    SELECT 社員番号, 社員名, 給料\\
    FROM 社員\\
    ORDER BY 給料 DESC\\
\end{flushleft}

\newpage
%----------------------------------------------------------4
\noindent
\section{社員表から,平均給料が25万円以下の部門番号と平均給料を出力せよ}

\begin{flushleft}
    SELECT 部門番号, AVG(給料) AS 平均給料\\
    FROM 社員\\
    GROUP BY 部門番号\\
    HAVING AVG(給料) $\leq$ 250000\\
\end{flushleft}

%----------------------------------------------------------5
\noindent
\section{社員表と部門から,社員番号,社員名,社員が所属する部門名を出力せよ}

\begin{flushleft}
    SELECT 社員番号, 社員名, 部門名\\
    FROM 社員, 部門\\
    WHERE 社員.部門番号 = 部門.部門番号\\
\end{flushleft}

%----------------------------------------------------------6
\noindent
\section{社員表から,部門番号が0100の社員の誰かと同額の給料の社員の社員
  番号と社員名を出力せよ}

\begin{flushleft}
    SELECT 社員番号, 社員名\\
    FROM 社員\\
    WHERE 給料 IN \\
    (SELECT 給料 FROM 社員 WHERE 部門番号=0100)\\
    AND 部門番号 != 0100 OR 部門番号 IS NULL\\
\end{flushleft}

\end{document}
%本文ここまで=========================================================