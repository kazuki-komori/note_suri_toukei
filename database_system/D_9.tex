\synctex=1
\documentclass[dvipdfmx,10pt, a4j]{jarticle}
%----------------------------------------------------------
%パッケージ読み込み
\usepackage{amsmath}
\usepackage{amssymb}
\usepackage{amsthm} %定理環境の拡張
\usepackage{ascmac}
\usepackage{bm}
\usepackage{cases}
\usepackage{comment} %非表示にするためのコメント
\usepackage{enumerate}
\usepackage{float} %画像をその場に表示.[h]の代わりに[H]
\usepackage{graphicx} % eps 形式の図版取り込みのため
\usepackage{mathrsfs}
\usepackage{url}
\usepackage[dvipdfmx]{hyperref}
\usepackage{color}
\usepackage{mathrsfs}
\usepackage[top=2cm, bottom=2cm, left=1cm, right=1cm]{geometry}
%----------------------------------------------------------

%----------------------------------------------------------
%命題関係の定義
\theoremstyle{definition}
\newtheorem{definition}{定義}[section]
\newtheorem{theorem}{定理}[section]
\newtheorem{proposition}[theorem]{命題}
\newtheorem{lemma}[theorem]{補題}
\newtheorem{col}[theorem]{系}
\newtheorem{example}{例}[section]
\newtheorem{remark}{注意}[section]
%----------------------------------------------------------

%タイトル・著者===================================================
\title{第9回 データベースシステム 課題}
\author{1108190116 \; 小森 一輝}
%=================================================================

%本文開始=========================================================
\begin{document}

\maketitle

%カウンタ--------------------------------------------------
\setcounter{section}{2}
%\setcounter{subsection}{0}
%\setcounter{subsubsection}{0}
%\setcounter{theorem}{0}

%----------------------------------------------------------定理2.8
\noindent
\textbf{1NF}
\begin{align*}
    \begin{tabular}{|c|c|c|c|c|c|c|c|c|c|c|c|c|c|c|}
        \hline
        SID&ID&店舗&店住所&日付&PID&商品名&単価&数量&小計&合計&税&請求額&PICID&PIC\\
        \hline
        \hline
        1&001&京都&京都市&20/11/01&ERS&消しゴム&100&2&200&1000&100&1100&KYO01&杉下\\
        \hline
        1&001&京都&京都市&20/11/01&SPN&シャーペン&300&1&300&1000&100&1100&KYO01&杉下\\
        \hline
        1&001&京都&京都市&20/11/01&STP&ホッチキス&500&1&500&1000&100&1100&KYO01&杉下\\
        \hline
        2&001&京都&京都市&20/11/02&SPN&シャーペン&300&2&600&1400&140&1540&KYO02&冠城\\
        \hline
        2&001&京都&京都市&20/11/02&BPN&ボールペン&200&4&800&1400&140&1540&KYO02&冠城\\
        \hline
        3&002&大阪&枚方市&20/11/03&STP&ホッチキス&500&1&500&2500&250&2750&OSK01&小出\\
        \hline
        3&002&大阪&枚方市&20/11/03&CAL&電卓&1000&2&2000&2500&250&2750&OSK01&小出\\
        \hline
    \end{tabular}
\end{align*}

主キーを \textbf{SID, 日付, ID, PID}に設定する. \\

\textbf{2NF}
\begin{align*}
    &\begin{tabular}{|c|c|c|c|c|c|c|c|c|}
        \hline
        \underline{SID}&\underline{ID}&\underline{日付}&\underline{PID}&数量&小計&合計&税&請求額\\
        \hline
        \hline
        1&001&20/11/01&ERS&2&200&1000&100&1100\\
        \hline
        1&001&20/11/01&SPN&1&300&1000&100&1100\\
        \hline
        1&001&20/11/01&STP&1&500&1000&100&1100\\
        \hline
        2&001&20/11/02&SPN&2&600&1400&140&1540\\
        \hline
        2&001&20/11/02&BPN&4&800&1400&140&1540\\
        \hline
        3&002&20/11/03&STP&1&500&2500&250&2750\\
        \hline
        3&002&20/11/03&CAL&2&2000&2500&250&2750\\
        \hline
    \end{tabular}
    &\begin{tabular}{|c|c|c|}
        \hline
        \underline{SID}&PICID&PIC\\
        \hline
        \hline
        1&KYO01&杉下\\
        \hline
        1&KYO01&杉下\\
        \hline
        1&KYO01&杉下\\
        \hline
        2&KYO02&冠城\\
        \hline
        2&KYO02&冠城\\
        \hline
        3&OSK01&小出\\
        \hline
        3&OSK01&小出\\
        \hline
    \end{tabular}\\
    &\begin{tabular}{|c|c|c|}
        \hline
        \underline{ID}&店舗&店住所\\
        \hline
        \hline
        001&京都&京都市\\
        \hline
        001&京都&京都市\\
        \hline
        001&京都&京都市\\
        \hline
        001&京都&京都市\\
        \hline
        001&京都&京都市\\
        \hline
        002&大阪&枚方市\\
        \hline
        002&大阪&枚方市\\
        \hline
    \end{tabular}
    &\begin{tabular}{|c|c|c|}
        \hline
        \underline{PID}&商品名&単価\\
        \hline
        \hline
        ERS&消しゴム&100\\
        \hline
        SPN&シャーペン&300\\
        \hline
        STP&ホッチキス&500\\
        \hline
        SPN&シャーペン&300\\
        \hline
        BPN&ボールペン&200\\
        \hline
        STP&ホッチキス&500\\
        \hline
        CAL&電卓&1000\\
        \hline
    \end{tabular}
\end{align*}

導出項目は削除した上で3NFは以下の通りになる.\\
\noindent
\textbf{3NF}
\begin{align*}
    &\begin{tabular}{|c|c|c|c|c|c|c|c|c|}
        \hline
        \underline{SID}&\underline{ID}&\underline{日付}&\underline{PID}&数量\\
        \hline
        \hline
        1&001&20/11/01&ERS&2\\
        \hline
        1&001&20/11/01&SPN&1\\
        \hline
        1&001&20/11/01&STP&1\\
        \hline
        2&001&20/11/02&SPN&2\\
        \hline
        2&001&20/11/02&BPN&4\\
        \hline
        3&002&20/11/03&STP&1\\
        \hline
        3&002&20/11/03&CAL&2\\
        \hline
    \end{tabular}\\
    &\begin{tabular}{|c|c|}
        \hline
        \underline{SID}&PICID\\
        \hline
        \hline
        1&KYO01\\
        \hline
        1&KYO01\\
        \hline
        1&KYO01\\
        \hline
        2&KYO02\\
        \hline
        2&KYO02\\
        \hline
        3&OSK01\\
        \hline
        3&OSK01\\
        \hline
    \end{tabular}
    &\begin{tabular}{|c|c|}
        \hline
        \underline{PICID}&PIC\\
        \hline
        \hline
        KYO01&杉下\\
        \hline
        KYO01&杉下\\
        \hline
        KYO01&杉下\\
        \hline
        KYO02&冠城\\
        \hline
        KYO02&冠城\\
        \hline
        OSK01&小出\\
        \hline
        OSK01&小出\\
        \hline
    \end{tabular}\\
    &\begin{tabular}{|c|c|}
        \hline
        \underline{ID}&店舗\\
        \hline
        \hline
        001&京都\\
        \hline
        001&京都\\
        \hline
        001&京都\\
        \hline
        001&京都\\
        \hline
        001&京都\\
        \hline
        002&大阪\\
        \hline
        002&大阪\\
        \hline
    \end{tabular}
    &\begin{tabular}{|c|c|}
        \hline
        \underline{店舗}&店住所\\
        \hline
        \hline
        京都&京都市\\
        \hline
        京都&京都市\\
        \hline
        京都&京都市\\
        \hline
        京都&京都市\\
        \hline
        京都&京都市\\
        \hline
        大阪&枚方市\\
        \hline
        大阪&枚方市\\
        \hline
    \end{tabular}\\    
    &\begin{tabular}{|c|c|c|}
        \hline
        \underline{PID}&商品名&単価\\
        \hline
        \hline
        ERS&消しゴム&100\\
        \hline
        SPN&シャーペン&300\\
        \hline
        STP&ホッチキス&500\\
        \hline
        SPN&シャーペン&300\\
        \hline
        BPN&ボールペン&200\\
        \hline
        STP&ホッチキス&500\\
        \hline
        CAL&電卓&1000\\
        \hline
    \end{tabular}
\end{align*}
%本文ここまで=========================================================
\end{document}