\synctex=1
\documentclass[dvipdfmx,10pt, a4j]{jarticle}
%----------------------------------------------------------
%パッケージ読み込み
\usepackage{amsmath}
\usepackage{amssymb}
\usepackage{amsthm} %定理環境の拡張
\usepackage{ascmac}
\usepackage{bm}
\usepackage{cases}
\usepackage{comment} %非表示にするためのコメント
\usepackage{enumerate}
\usepackage{float} %画像をその場に表示.[h]の代わりに[H]
\usepackage{graphicx} % eps 形式の図版取り込みのため
\usepackage{mathrsfs}
\usepackage{url}
\usepackage[dvipdfmx]{hyperref}
%----------------------------------------------------------

%----------------------------------------------------------
%命題関係の定義
\theoremstyle{definition}
\newtheorem{definition}{定義}[section]
\newtheorem{theorem}{定理}[section]
\newtheorem{proposition}[theorem]{命題}
\newtheorem{lemma}[theorem]{補題}
\newtheorem{col}[theorem]{系}
\newtheorem{example}{例}[section]
\newtheorem{remark}{注意}[section]
%----------------------------------------------------------

%タイトル・著者===================================================
\title{第2回 データベースシステム 課題}
\author{小森 一輝}
%=================================================================

%本文開始=========================================================
\begin{document}

    \maketitle

%カウンタ--------------------------------------------------
    \setcounter{section}{2}
%\setcounter{subsection}{0}
%\setcounter{subsubsection}{0}
%\setcounter{theorem}{0}
%----------------------------------------------------------問1
    \noindent
    \textbf{問1} 積和標準形\\
    \begin{align*}
        (A\, IMP\, B)\, OR\, (B\, IMP\, A)
        &=\, (NOT\, A\, OR\, B)\, OR\, (NOT\, B\, OR\, A)\\
        &=\, NOT\, A\, OR\, B\, OR\, NOT\, B\, OR\, A\\
    \end{align*}
    \begin{align*}
        (A\, EQV\, B)\, AND\, (B\, EQV\, A)
        &=\, (NOT\, A\, AND\, NOT\, B\, OR\, A\, AND\, B)\\
        &AND\, (NOT\, B\, AND\, NOT\, A\, OR\, B\, AND\, A)\\
        交換法則より、2つの EQV は同値であるから、\\
        &= NOT\, A\, AND\, NOT\, B\, OR\, A\, AND\, B\\
    \end{align*}
%----------------------------------------------------------問2.1
    \noindent
    \textbf{問2.1.} 積和標準形X\\
    \begin{align*}
        &A\, AND\, B\, AND\, C\, OR\\
        &A\, AND\, NOT\, B\, AND\, NOT\, C\, OR\\
        &NOT\, A\, AND\, NOT\, B\, AND\, C\, OR\\
        &NOT\, A\, AND\, NOT\, B\, AND\, NOT\, C\\
    \end{align*}
%----------------------------------------------------------問2.2
    \noindent
    \textbf{問2.2.} 和積標準形Y\\
    \begin{align*}
        &NOT\, A\, OR\, NOT\, B\, OR\, C\, AND\\
        &NOT\, A\, OR\, B\, OR\, NOT\, C\, AND\\
        &A\, OR\, NOT\, B\, OR\, NOT\, C\, AND\\
        &A\, OR\, NOT\, B\, OR\, C\\
    \end{align*}

    \newpage
%----------------------------------------------------------問2.3
    \noindent
    \textbf{問2.3.} X\,と\, Yが等しいことの証明\\

    それぞれの場合について、真理値表は以下のように表される。\\

    \begin{tabular}{|c|c|c|c|c|c|c|c|c|}
        \hline
        A & B & C & A\, AND\, B & A\, OR\, B & B\, AND\, C & B\, OR\, C & C\, AND\, A & C\, OR\, A\\
        \hline
        \hline
        T & T & T & T & T & T & T & T & T\\
        \hline
        T & F & T & F & T & F & T & T & T\\
        \hline
        T & F & F & F & T & F & F & F & T\\
        \hline
        F & T & F & F & T & F & T & F & F\\
        \hline
        F & F & T & F & F & F & T & F & T\\
        \hline
        F & F & F & F & F & F & F & F & F\\
        \hline
    \end{tabular}\\

    まず、真理値表より、Xを整理する。\\

    \begin{tabular}{|c|c|c|c|}
        \hline
        A & B & C & X\\
        \hline
        \hline
        T & T & T & T\\
        \hline
        T & F & T & F\\
        \hline
        T & F & F & T\\
        \hline
        F & T & F & F\\
        \hline
        F & F & T & T\\
        \hline
        F & F & F & T\\
        \hline
    \end{tabular}\\

    次に、真理値表より、Yを整理する。\\

    \begin{tabular}{|c|c|c|c|}
        \hline
        A & B & C & Y\\
        \hline
        \hline
        T & T & T & T\\
        \hline
        T & F & T & F\\
        \hline
        T & F & F & T\\
        \hline
        F & T & F & F\\
        \hline
        F & F & T & T\\
        \hline
        F & F & F & T\\
        \hline
    \end{tabular}\\

    以上の2つの真理値表より、XとYの真理値は一致することから、XとYは等しいことが示された。
\end{document}
%本文ここまで=========================================================
