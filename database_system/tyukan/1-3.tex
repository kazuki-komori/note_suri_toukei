\synctex=1
\documentclass[dvipdfmx,10pt, a4j]{jarticle}
%----------------------------------------------------------
%パッケージ読み込み
\usepackage{amsmath}
\usepackage{amssymb}
\usepackage{amsthm} %定理環境の拡張
\usepackage{ascmac}
\usepackage{bm}
\usepackage{cases}
\usepackage{comment} %非表示にするためのコメント
\usepackage{enumerate}
\usepackage{float} %画像をその場に表示.[h]の代わりに[H]
\usepackage{graphicx} % eps 形式の図版取り込みのため
\usepackage{mathrsfs}
\usepackage{url}
\usepackage[dvipdfmx]{hyperref}
%----------------------------------------------------------

%----------------------------------------------------------
%命題関係の定義
\theoremstyle{definition}
\newtheorem{definition}{定義}[section]
\newtheorem{theorem}{定理}[section]
\newtheorem{proposition}[theorem]{命題}
\newtheorem{lemma}[theorem]{補題}
\newtheorem{col}[theorem]{系}
\newtheorem{example}{例}[section]
\newtheorem{remark}{注意}[section]
%----------------------------------------------------------

%タイトル・著者===================================================
\title{データベースシステム 中間確認1-3}
\author{1108190116 \, 小森 一輝}
%=================================================================

%本文開始=========================================================
\begin{document}

    \maketitle

%カウンタ--------------------------------------------------
    \setcounter{section}{2}
%\setcounter{subsection}{0}
%\setcounter{subsubsection}{0}
%\setcounter{theorem}{0}
%----------------------------------------------------------問1
    \noindent
    \textbf{中間確認1-3} $(\forall_x)((f(x)\; OR\; (\exists_x)(f(x)\; IMP\; g(x))\; IMP\; f(x)\; OR\; (\exists_x)(g(x)\; IMP\; NOT\; g(y))))$\\
    \begin{align*}
        &(\forall_x)(f(x)\; OR\; (\exists_x)((f(x)\; IMP\; g(x))\; IMP\; f(x)\; OR\; (\exists_x)(g(x)\; IMP\; NOT\; f(x))))\\
        &= (\forall_x)(f(x)\; OR\; (\exists_x)((f(x)\; IMP\; g(x))\; IMP\; f(x)\; OR\; (\exists_x)(NOT\; g(x)\; OR\; NOT\; g(y))))\\
        &= (\forall_x)(f(x)\; OR\; (\exists_x)((f(x)\; IMP\; g(x))\; IMP\; f(x)\; OR\; (\exists_x)(NOT\; g(x)\; OR\; NOT\; g(y))))\\
        &= (\forall_x)(f(x)\; OR\; (\exists_x)((f(x)\; IMP\; g(x))\; IMP\; f(x)\; OR\; (NOT\; (\exists_x)\; (g(x))\; OR\; NOT\; (\exists_x)\; (g(y))))) \\ &\qquad (\because 述語論理式の基本公式)\\
        &= (\forall_x)(f(x)\; OR\; (\exists_x)((NOT\; f(x)\; OR\; g(x))\; IMP\; f(x)\; OR\; (NOT\; (\exists_x)\; (g(x))\; OR\; NOT\; (\exists_x)\; (g(y))))) \\
        &= (\forall_x)(f(x)\; OR\; (\exists_x)(NOT(NOT\; f(x)\; OR\; g(x))\; OR\; f(x)\; OR\; (NOT\; (\exists_x)\; (g(x))\; OR\; NOT\; (\exists_x)\; (g(y)))))\\ 
        &= (\forall_x)(f(x)\; OR\; (\exists_x)(NOT(NOT\; f(x))\; AND\; NOT\; g(x)\; OR\; f(x)\; OR\; (NOT\; (\exists_x)\; (g(x))\; OR\; NOT\; (\exists_x)\; (g(y)))))\\ &\qquad (\because ド・モルガンの公式)\\
        &= (\forall_x)(f(x)\; OR\; (\exists_x)(f(x)\; AND\; NOT\; g(x)\; OR\; f(x)\; OR\; NOT\; (\exists_x)\; (g(x))\; OR\; NOT\; (\exists_x)\; (g(y))))\\ &\qquad (\because 二重否定)\\
        &= (\forall_x)(f(x)\; OR\; (\exists_x)(f(x)\; AND\; NOT\; g(x)\; OR\; f(x)\; OR\; NOT\; (\exists_x)\; (g(x))\; OR\; NOT\; (\exists_x)\; (g(x))))\\ &\qquad (\because 述語論理式の基本公式)\\
        &= (\forall_x)(f(x)\; OR\; (\exists_x)(f(x)\; AND\; NOT\; g(x)\; OR\; f(x)\; OR\; NOT\; (\exists_x)\; (g(x))))\\ &\qquad (\because 同一律)\\
        &= (\forall_x)(f(x)\; OR\; (\exists_x)(f(x)\; AND\; NOT\; g(x)\; OR\; NOT\; (\exists_x)\; (g(x))))\\ &\qquad (\because 同一律)\\
        &= (\forall_x)(f(x)\; OR\; (\exists_x)(f(x)\; AND\; NOT\; g(x)))\\ &\qquad (\because 同一律)\\
        &= (\forall_x)(f(x)\; OR\; (\forall_x)(f(x))\; AND\; NOT\;(\forall_x)(g(x)) \qquad (\because 述語論理式の基本公式)\\
        &= (\forall_x)(f(x)\; AND\; NOT\;(\forall_x)(g(x)) \qquad (\because 同一律)\\
        &= (\forall_x)(f(x)\; AND\; NOT\; g(x))\\
    \end{align*}
    \end{document}
%本文ここまで=========================================================
