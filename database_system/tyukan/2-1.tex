\synctex=1
\documentclass[dvipdfmx,10pt, a4j]{jarticle}
%----------------------------------------------------------
%パッケージ読み込み
\usepackage{amsmath}
\usepackage{amssymb}
\usepackage{amsthm} %定理環境の拡張
\usepackage{ascmac}
\usepackage{bm}
\usepackage{cases}
\usepackage{comment} %非表示にするためのコメント
\usepackage{enumerate}
\usepackage{float} %画像をその場に表示.[h]の代わりに[H]
\usepackage{graphicx} % eps 形式の図版取り込みのため
\usepackage{mathrsfs}
\usepackage{url}
\usepackage[dvipdfmx]{hyperref}
\usepackage{color}
\usepackage{mathrsfs}
\usepackage[top=2cm, bottom=2cm, left=1cm, right=1cm]{geometry}
%----------------------------------------------------------

%----------------------------------------------------------
%命題関係の定義
\theoremstyle{definition}
\newtheorem{definition}{定義}[section]
\newtheorem{theorem}{定理}[section]
\newtheorem{proposition}[theorem]{命題}
\newtheorem{lemma}[theorem]{補題}
\newtheorem{col}[theorem]{系}
\newtheorem{example}{例}[section]
\newtheorem{remark}{注意}[section]
%----------------------------------------------------------

%タイトル・著者===================================================
\title{データベースシステム 中間確認2-1}
\author{1108190116 \, 小森 一輝}
%=================================================================

%本文開始=========================================================
\begin{document}

\maketitle

%カウンタ--------------------------------------------------
\setcounter{section}{2}
%\setcounter{subsection}{0}
%\setcounter{subsubsection}{0}
%\setcounter{theorem}{0}

%----------------------------------------------------------定理2.8
\noindent
\textbf{中間確認2-1}\\
\textbf{R1}
\begin{align*}
    \begin{tabular}{|c|c|c|c|c|c|c|c|c|c|}
        \hline
        \underline{CID}&顧客名&\underline{購入日}&\underline{ISBN}&商品名&数量&単価&合計金額&JID&ジャンル\\
        \hline
        \hline
        2001&山田哲人&2020/12/1&4873119049&プログラミングTypeScript&2&3740&7480&100&ソフト開発\\
        \hline
        2001&山田哲人&2020/12/2&4297100916&Vue.js入門&1&3718&3718&101&IT\\
        \hline
        2002&鈴木一郎&2020/12/2&4297100916&Vue.js入門&2&3718&7436&101&IT\\
        \hline
        2002&鈴木一郎&2020/12/2&479739739X&AWS認定資格試験テキスト&3&3780&11340&101&IT\\
        \hline
        2003&田中将大&2020/12/4&4048930656&Clean Architecture&1&10128&10128&101&IT\\
        \hline
        2003&田中将大&2020/12/4&4873119049&プログラミングTypeScript&1&3740&3740&100&ソフト開発\\
        \hline
        2004&大谷翔平&2020/12/3&4865941959&GCPの教科書&4&3960&15840&101&IT\\
        \hline
    \end{tabular}
\end{align*}

主キーを \textbf{CID, 購入日, ISBN}に設定する. 尚, 書籍情報はAmazonに出品されているデータを参考にしている.\\

\end{document}
%本文ここまで=========================================================